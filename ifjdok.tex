\documentclass{article}
\usepackage[utf8]{inputenc}
\author{}

\title{Dokumentácia k projektu IFJ/IAL
\textbf{Implementácia interpretu imperativného jazyka IFJ17} 
}
\usepackage{natbib}
\usepackage{graphicx}

\begin{document}

\begin{figure}[h!]
\centering
\includegraphics{FITlogo.jpg}
\maketitle
\end{figure}

\vspace{3cm}

\noindent
\large{\textbf{Tím číslo: 94}}
\\
\large{\textbf{Varianta: II}}
\\
\large{\textbf{Autori:}} 
\\
Tomáš Nereča \\
Samuel Obuch \\
Jiří Vozár \\
Ján Farský \\

\vspace{7cm}

\textbf{ \Huge{Obsah} }

\begin{enumerate}
    \item \LARGE{Úvod} 
    \item \LARGE{Zadanie} 
        \begin{itemize}
            \item Varianta zadania
        \end{itemize}
    \item \LARGE{Implementácia} 
        \begin{itemize}
            \item Implementácia
            \item Lexikálna analýza
            \item Algoritmy lexikálnej analýzy
            \item Syntaktická analýza
            \item Algoritmy syntaktickej analýzy
        \end{itemize}
    \item \LARGE{Rozšírenia} 
    \item \LARGE{Testovanie} 
    \item \LARGE{Práca v tíme} 
    \item \LARGE{Záver} 
    \item \LARGE{Prílohy} 
\end{enumerate}

\vspace{8cm}

\section{Úvod}
Dokumentácia popisuje implementáciu imperatívneho prekladača jazyku IFJ17. Program prekladaču
 najprv načíta zdrojový súbor, potom ho vyhodnotí z hľadiska syntaxe a ak je všetko v poriadku, 
 kód interpretuje. V prípade nájdenia chyby nasleduje výpis chybového hlásenia. 
Dokumentácia taktiež zahrňuje prácu v tíme a prílohy približujúce logiku fungovania jednotlivých 
častí prekladača. 

\section{Zadanie}
Jazyk IFJ17 je podmnožinou jazyka Free Basic...

\subsection{Varianta zadania}

\section{Implementácia}

\subsection{Implementácia}
\subsection{Lexikálna analýza}
\subsection{Algoritmy lexikálnej analýzy}
\subsection{Syntaktická analýza}
\subsection{Algoritmy syntaktickej analýzy}

\section{Rozšírenia}

\section{Testovanie}

\section{Práca v tíme}

\section{Záver}

\section{Prílohy}

\end{document}
