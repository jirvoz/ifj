\documentclass{article}
\usepackage[slovak]{babel}
\usepackage[utf8]{inputenc}

\usepackage{indentfirst}
\usepackage{graphicx}

\begin{document}
    \begin{titlepage}
        \begin{center}
            \vspace{\stretch{0.382}}
            
            \includegraphics{FITlogo.jpg}\\[16mm]
            
            \LARGE
            Implementácia interpretu imperativného jazyka IFJ17 \\
            \large
            Dokumentácia k projektu IFJ/IAL\\[8mm]
            
            \begin{tabular}{r l}
                Tým & 094 \\
                Varianta & II \\
                Vedoucí: & Tomáš Nereča (xnerec00) \quad 26\% \\
                Další členové & Samuel Obuch (xobuch00) \quad 26\% \\
                    & Jiří Vozár (xvozar04) \quad 26\% \\
                    & Ján Farský (xfarsk00) \quad 22\% \\
                Implementované rozšíření & BASE, FUNEXP, IFTHEN\\
            \end{tabular}
            
            \vspace{\stretch{0.618}}
        \end{center}
    \end{titlepage}

    \tableofcontents
    \newpage
    
    \section{Úvod}
        Dokumentácia popisuje implementáciu imperatívneho prekladača jazyku IFJ17. Program prekladaču
        najprv načíta zdrojový súbor, potom ho vyhodnotí z hľadiska syntaxe a ak je všetko v poriadku, 
        kód interpretuje. V prípade nájdenia chyby nasleduje výpis chybového hlásenia. 
        Dokumentácia taktiež zahrňuje prácu v tíme a prílohy približujúce logiku fungovania jednotlivých 
        častí prekladača. 
        
    \subsection{Varianta zadania}
    
    \section{Implementácia}
    
    \subsection{Implementácia}
    \subsection{Lexikálna analýza}
    \subsection{Algoritmy lexikálnej analýzy}
    \subsection{Syntaktická analýza}
    \subsection{Algoritmy syntaktickej analýzy}
    
    \section{Rozšírenia}
    
    \section{Testovanie}
    
    \section{Práca v tíme}
    
    \section{Záver}
    
    \section{Prílohy}

\end{document}
