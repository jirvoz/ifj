\documentclass{article}
\usepackage[slovak]{babel}
\usepackage[utf8]{inputenc}

\usepackage{indentfirst}
\usepackage{graphicx}
\usepackage{lscape}

\begin{document}
    \begin{titlepage}
        \begin{center}
            \vspace{\stretch{0.382}}
            
            \includegraphics[width=10cm]{FITlogo.png}\\[30mm]
            
            \LARGE
            Implementácia prekladača imperatívneho jazyka IFJ17 \\
            \large
            Dokumentácia k~projektu IFJ/IAL\\[55mm]
            
            \begin{tabular}{r l l}
                Tím: & 094 & \\
                Varianta: & II & \\
                Vedúci: & Tomáš Nereča (xnerec00) & 26\% \\
                Ďalší členovia: & Samuel Obuch (xobuch00) & 26\% \\
                    & Jiří Vozár (xvozar04) & 26\% \\
                    & Ján Farský (xfarsk00) & 22\% \\
                Implementované rozšírenia: & BASE, FUNEXP, IFTHEN &\\
            \end{tabular}
            
            \vspace{\stretch{0.618}}
        \end{center}
    \end{titlepage}

    \tableofcontents
    \newpage
    
    \section{Úvod}
        Dokumentácia popisuje implementáciu prekladača imperatívneho jazyka IFJ17. Prekladač načítava
        zdrojový kód jazyka IFJ17 zo~štandardného vstupu a ten pomocou lexikálnej, syntaktickej a sémantickej
        analýzy vyhodnotí a vygeneruje potrebné inštrukcie v~IFJcode17 pre interpret. V~prípade
        výskytu chyby pri analýze kódu IFJ17 prekladač vráti kód príslušnej chyby a chybovú hlášku na štandardný
        chybový výstup.
        
        Dokumentácia taktiež zahŕňa zhodnotenie práce v~tíme a približuje logiku a fungovanie
        jednotlivých častí prekladača. 
    
    \section{Implementácia}
    
        \subsection{Lexikálna analýza}
            Hlavná funkcia lexikálneho analyzátora je \emph{getNextToken}. Je implementovaná v súbore \emph{scanner.c}.
            Jej argumenty sú ukazatele na štruktúru typu \emph{tToken} a \emph{FILE} a jej návratový typ je \emph{bool}. 

            Funkcia číta lexémy zo zadaného súboru(spravidla \emph{stdin}). Lexémy sú spracované pomocou konečného automatu. 
            Ak sa automat dostane do konečného stavu, funkcia vráti hodnotu \emph{true} a podľa spracovaných lexém nastaví
            token predaný v argumente. Ak nastane počas lexikálnej analýzy chyba, vráti funkcia hodnotu \emph{false} a 
            token nenastavuje. 

            Dátová štruktúra \emph{tToken} a jej podštruktúry sú definované v hlavičkovom súbore \emph{scanner.h}. 
            Platný token musí mať vždy definovaný typ, a ak to typ tokenu vyžaduje tak aj atribút 
            (napr. načítaný reťazec v prípade, že je token typu \emph{string}). 

            Funkcia \emph{getNextToken} sa taktiež stará o prevod veľkých písmen na malé a počítanie riadkov.
            Taktiež uľahčuje prácu syntaktickému analyzátoru, pretože rozlišuje medzi identifikátormi a kľučovými slovami.
            V poli reťazcov sú uložené všetky kľúčové slová a \emph{binárnym vyhľadávaním} vo funkcii \emph{identifierTest} 
            sa testuje, či sa jedná o identifikátor alebo kľúčové slovo.

            Diagram konečného automatu nájdete v prílohe.

        \subsection{Syntaktická analýza}
            Překladač využívá syntaxí řízený překlad, ve kterém kombinuje \emph{rekurzivní sestup} pro základní konstrukce
            a příkazy jazyka s \emph{precedenční syntaktickou analýzou} pro výrazy.
            Vstupním bodem překladače je funkce \texttt{parse}, která zavolá vyhodnocení hlavního neterminálu \emph{program},
            který po kontrole volá funkce z něho vycházejících dalších neterminálů.
            Pokud pravidlo očekává výraz, je volána funkce \texttt{expression}, která se postará o vyhodnocení výrazu převodem
            do postfixové podoby.
            
            \subsubsection{Rekurzívny zostup}
                Pro každý neterminál je vytvořena funkce, která je volána pro vyhodnocení daného neterminálu.
                Funkce poté vrací hodnotu \texttt{true}, pokud bylo vyhodnocení úspěšné a hodnotu \texttt{false}, pokud došlo k nějaké chybě. Některé funkce z důvodu zjednodušení nereprezentují pravidla úplně přesně, kdy se například funkce \texttt{statementList} nevolá rekurzivně, ale je řízena cyklem. Samotný sestup je rozdělen do 3 modulů pro oddělení obecných neterminálů, neterminálů příkazů a neterminálů funkcí.
            
            \subsubsection{Precedenčná syntaktická analýza}
    
        \subsection{Tabuľka symbolov}
            Tabulka symbolů je implementována jako hash tabulka. Hashovací funkce byla zvolena \emph{djb2}\footnote{\texttt{http://www.cse.yorku.ca/\~{}oz/hash.html}}, jelikož dávala dobré výsledky rozložení při experimentování s větším množstvím záznamů.
            
            Velikost pole pro ukazatele byla zvolena $128$ aby se operace modulo v hash funkci převedla na bitové maskování a pro běžný počet proměnných ve funkci je dostačující.
            
            Informace o symbolu jsou poté ukládány ve struktuře, která byla navržená univerzálně pro informace o proměnných i funkcích.
        
        \subsection{Vstavané funkcie}
            V prekladači sme implementovali aj štyri vstavané funkcie jazyka IFJ17 a to \textbf{Length, Chr, Asc} a \textbf{Substr}.
            Funkcie sme implementovali v module ifunc. Ak funkcia \texttt{expression} pri vyhodnocovaní výrazu narazí na volanie 
            niektorej zo vstavaných funkcií, zavolá si konkrétnu funkciu z modulu ifunc. Tá si následne spracuje argumenty a následne
            vygeneruje potrebné inštrukcie. Vstavané funkcie pracujú s premennými na globálnom rámci, ktoré sú vytvorené pri štarte programu.

            \paragraph{Length}
            Pomocou inštrukcie \emph{STRLEN} sa získa dĺžka stringu(argument funkcie). Je potrebné použiť pomocné premenné, pretože 
            inštrukcia \emph{STRLEN} nemá zásobníkovú variantu.

            \paragraph{Chr}
            Pomocou zásobníkovej inštrukcie \emph{INT2CHARS} sa jednoducho prevedie zadané číslo z ascii tabuľky na znak.

            \paragraph{Asc}
            Po kontrole argumentov pomocou inštrukcií \emph{GT} a \emph{LT} sa buď skončí na náveštie pomocou inštrukcie 
            \emph{JUMP} (v tomto prípade vráti funkcia hodnotu 0), alebo sa vykoná inštrukcia \emph{STRI2INT},
            ktorá prevedie zadaný znak na požadovanom indexe na číslo z ascii tabuľky.

            \paragraph{SubStr}
            Riešenie tejto funkcie bolo zložitejšie, pretože pre získanie podreťazca sa v inštrukčnej sade nenachádza jednoduchá
            inštrukcia. Po kontrole argumentov podobne ako pre funkciu \textbf{Asc}, sa v cykle tvoril výstupný reťazec
            pomocou konkatenácie vždy jedného znaku na koniec výstupného reťazca. K tomu sme využili inštrukcie ako napríklad
            \emph{ADD}, \emph{JUMPIFEQ} alebo \emph{GETCHAR}.
            
        \subsection{Generovanie kódu}
            Cílový kód je generován přímo během jediného průchodu vypisováním příslušných instrukcí ve funkci
            příslušného pravidla.
            
            Proměnné definované uživatelem se ukládají na datový rámec, který se pro funkce
            tvoří vždy nový.
            Výrazy využívají pro mezivýsledky zásobník, kam se ukládají i vyhodnocené parametry a výsledky funkcí.
            
            Podmínky a cykly využívají instrukce podmíněných a nepodmíněných skoků, ale funkce se volají instrukcemi \texttt{CALL}
            a \texttt{RETURN}.
        
    \section{Rozšírenia}
    Boli implementované celkom 3 rozšírenia
        \begin{itemize}
            \item \textbf{base}   - Lexikálny analyzátor dokáže prečítať celé číslo zadané v inej než
                                    desiatkovej sústave a následne ho previesť pomocou funkcie 
                                    \emph{strtol} s argumentom príslušnej číselnej sústavy do \emph{integeru}.
            \item \textbf{funexp} - Díky vhodnému návrhu nezávislého vyhodnocování výrazů a předávání parametrů funkcí
                                    přes zásobník jako výsledky rekurzivně volaných podvýrazů bylo toto rozšíření implementováno
                                    už v základní verzi překladače.
            \item \textbf{ifthen} - Podpora \emph{elseif} a možnost vynechání \emph{else} obnášela pouze lehkou úpravu funkce
                                    pro vyhodnocování podmínek přepsáním do cyklu a rozšířením generování instrukcí
                                    pro skoky a návěští.
        \end{itemize}
    
    \section{Testovanie}
    Na začiatku sme určili jedného člena, ktorý písal jednotkové testy pre lexikálny analyzátor. 
    Neskôr sme napísali zopár vlastných regresných testov.
    
    Keď sme sa dozvedeli o~verejnej databáze testov našich kolegov. Do nej sme prispeli aj vlastnými testami a využili ostatné testy na čo najrobustnejšie otestovanie funkčnosti nášho prekladača.
    
    \section{Práca v~tíme}
    Ako tím sme sa začali schádzať prakticky ihneď po zaregistrovaní nášho zadania. Stretávali sme sa 
    väčšinou raz do týždňa. Vždy sme prediskutovali aktuálny stav práve implementovaných častí 
    a ďalší postup či korekcie v~zdrojovom kóde. Prácu sme sa snažili rozdeľovať rovnomerne medzi 
    všetkých členov tímu, čo sa nám nepodarilo vždy. Po pridelení úlohy sme stanovili deadline, aby sme 
    mohli čo najskôr pokračovať na nasledujúcej časti projektu. Jednotlivé časti sme sa snažili 
    implementovať paralelne s~preberanou látkou na prednáškach, aby sme sa vyhli chybám z dôvodu neznalosti.

        \subsection{Správa zdrojového kódu}
        Pre správu a zdielanie zdrojových súborov sme využili verzovací systém \emph{Git} a webovú službu GitHub pre vzdialené ukladanie, ktorú sme už počas štúdia využili na verzovanie projektov v~iných predmetoch.
        
        Ako komunikačný kanál sme využívali prevažne skupinovú konverzáciu na sociálnej sieti Facebook, ktorá nám všetkým 
        vyhovovala.
        
        Na sledovanie postupu pri plnení pridelených úloh a oznamovanie nájdených chýb 
        sme využívali nástroj Trello, ktorý slúži ako online kanban pre sledovanie projektov. 
        
        Vďaka týmto informačným kanálom mohli mať všetci členovia tímu prístup k~najaktuálnejšej
        verzii projektu a reagovať na vzniknuté chyby efektívne.

        \subsection{Rozdelenie práce v tíme}

        \textbf{Tomáš Nereča} pracoval na lexikálnom analyzátore a v neskoršej fáze vývoja pomáhal pri implementácii niektorých
        funkcií precedenčnej syntaktickej analýzy. Okrem toho sa podieľal na menších moduloch errors, stack a ifunc. 

        \textbf{Jiří Vozár} měl na starosti vedení implementace, návrh komunikace mezi moduly a implementaci syntaktické analýzy rekurzivním sestupem.

        Dôvodom nerovnomerného rozdelenia bodov bolo nedostatočné splnenie úloh jedného 
        z~členov tímu, čo viedlo k nutnosti zapojiť k týmto úlohám iných členov tímu.

    \section{Záver}
    S~projektom podobného rozsahu sa ešte nikto z~nás predtým nestretol, preto ho považujeme za dobrú
    skúsenosť pre každého z~nás. Pri jeho riešení sme prakticky využili získané vedomosti z~predmetov 
    IFJ a IAL.
    
    Pre správne fungovanie tímu bolo potrebné kvalitné riadenie a pridelovanie úloh a pravideľná
    komunikácia medzi jednotlivými členmi tímu. Výsledkom tejto práce je funkčný a z~nášho pohľadu vydarený 
    prekladač jazyka IFJ17.
    
    \newpage
    \section{Prílohy}
        \subsection{Diagram konečného automatu}
            \includegraphics[trim=6cm 0 0 0, width=15cm]{finite_automata.png}
            \newpage
            
        \subsection{LL gramatika}
            \begin{enumerate}
                % NESAHAT - zhenodnotí čísla v tabulce LL gramatiky!
                \item \texttt{<program> -> declare function <functionDecl> eol <program>}
                \item \texttt{<program> -> function <functionDef> eol <program>}
                \item \texttt{<program> -> scope <statementList> end scope}
                
                \item \texttt{<functionDecl> -> <functionHeader>}
                \item \texttt{<functionDef> -> <functionHeader> eol <statementList> end function}
                
                \item \texttt{<functionHeader> -> identifier ( <functionParams> ) as type}
                
                \item \texttt{<functionParams> -> <functionParam> <nextFuncParam>}
                \item \texttt{<functionParams> -> epsilon}
                
                \item \texttt{<nextFuncParam> -> , <functionParam> <nextFuncParam>}
                \item \texttt{<nextFuncParam> -> epsilon}
                
                \item \texttt{<functionParam> -> identifier as type}
                
                \item \texttt{<statementList> -> <statement> eol <statementList>}
                \item \texttt{<statementList> -> epsilon}
                
                \item \texttt{<statement> -> dim <declaration>}
                \item \texttt{<statement> -> identifier = <expression>}
                \item \texttt{<statement> -> input identifier}
                \item \texttt{<statement> -> print <printArgs>}
                \item \texttt{<statement> -> if <expression> then eol <statementList> <else>}
                \item \texttt{<statement> -> do while <expression> eol <statementList> loop}
                \item \texttt{<statement> -> return identifier}
                
                \item \texttt{<declaration> -> identifier as type}
                \item \texttt{<declaration> -> identifier as type = <expression>}
                
                \item \texttt{<printArgs> -> <expression> ;}
                \item \texttt{<printArgs> -> <expression> ; <printArgs>}
                
                \item \texttt{<else> -> elseif <expression> then eol <else> end if}
                \item \texttt{<else> -> else eol <statementList> end if}
                \item \texttt{<else> -> end if}
                
                \item \texttt{<expression> ->} vyhodnocuje se precedenční syntaktickou analýzou
            \end{enumerate}
        \newpage
        
        \subsection{LL tabuľka}
        \newcommand{\tterm}[1]{\rotatebox[origin=c]{90}{\texttt{#1}}}
            \begin{tabular}{|r|*{10}{c|}}
                \hline
                & \tterm{declare} & \tterm{function} & \tterm{scope} & \tterm{identifier} & \tterm{dim} &
                \tterm{input} & \tterm{print} & \tterm{if} & \tterm{do} & \tterm{return} \\\hline \hline
                \texttt{<program>} & 1 & 2 & 3 &&&&&&& \\\hline
                \texttt{<functionDecl>} &&&& 4 &&&&&& \\\hline
                \texttt{<functionDef>} &&&& 5 &&&&&& \\\hline
                \texttt{<functionHeader>} &&&& 6 &&&&&& \\\hline
                \texttt{<functionParams>} &&&& 7, 8 &&&&&& \\\hline
                \texttt{<nextFuncParam>} &&&&&&&&&& \\\hline
                \texttt{<functionParam>} &&&& 11 &&&&&& \\\hline
                \texttt{<statementList>} &&&& 12 & 12 & 12 & 12 & 12 & 12 & 12 \\\hline
                \texttt{<statement>} &&&& 15 & 14 & 16 & 17 & 18 & 19 & 20 \\\hline
                \texttt{<declaration>} &&&& 21, 22&&&&&& \\\hline
                \texttt{<else>} &&&& 23, 24&&&&&& \\\hline
            \end{tabular}
            
            \begin{tabular}{|r|*{9}{c|}}
                \hline
                & \tterm{elseif} & \tterm{else} & \tterm{end} & \tterm{loop} & \tterm{eol} &
                \tterm{(} & \tterm{)} & \tterm{=} & \tterm{,} \\\hline \hline
                \texttt{<program>} &&&&&&&&& \\\hline
                \texttt{<functionDecl>} &&&&&&&&& \\\hline
                \texttt{<functionDef>} &&&&&&&&& \\\hline
                \texttt{<functionHeader>} &&&&&&&&& \\\hline
                \texttt{<functionParams>} &&&&&&& 8 && \\\hline
                \texttt{<nextFuncParam>} &&&&&&& 10 && 9 \\\hline
                \texttt{<statementList>} & 13 & 13 & 13 & 13 &&&&& \\\hline
                \texttt{<statement>} &&&&&&&&& \\\hline
                \texttt{<declaration>} &&&&&& 23, 24&&& \\\hline
                \texttt{<else>} & 25 & 26 & 27 &&&&&& \\\hline
            \end{tabular}
        \newpage

        \subsection{Precedenčná tabuľka}

\end{document}