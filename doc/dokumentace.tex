\documentclass{article}
\usepackage[slovak]{babel}
\usepackage[utf8]{inputenc}

\usepackage{indentfirst}
\usepackage{graphicx}

\begin{document}
    \begin{titlepage}
        \begin{center}
            \vspace{\stretch{0.382}}
            
            \includegraphics{FITlogo.jpg}\\[16mm]
            
            \LARGE
            Implementácia interpretu imperativného jazyka IFJ17 \\
            \large
            Dokumentácia k projektu IFJ/IAL\\[8mm]
            
            \begin{tabular}{r l}
                Tím & 094 \\
                Varianta & II \\
                Vedúci: & Tomáš Nereča (xnerec00) \quad 26\% \\
                Ďalší členovia & Samuel Obuch (xobuch00) \quad 26\% \\
                    & Jiří Vozár (xvozar04) \quad 26\% \\
                    & Ján Farský (xfarsk00) \quad 22\% \\
                Implementované rozšírenia & BASE, FUNEXP, IFTHEN\\
            \end{tabular}
            
            \vspace{\stretch{0.618}}
        \end{center}
    \end{titlepage}

    \tableofcontents
    \newpage
    
    \section{Úvod}
        Dokumentácia popisuje implementáciu imperatívneho prekladača jazyku IFJ17. Program prekladaču
        najprv načíta zdrojový súbor, potom ho vyhodnotí z hľadiska syntaxe a ak je všetko v poriadku, 
        kód interpretuje. V prípade nájdenia chyby nasleduje výpis chybového hlásenia. 
        Dokumentácia taktiež zahrňuje prácu v tíme a prílohy približujúce logiku fungovania jednotlivých 
        častí prekladača. 
        
        \subsection{Varianta zadania}
    
    \section{Implementácia}
    
        \subsection{Lexikálna analýza}
        
        \subsection{Syntaktická analýza}
            \subsubsection{Rekurzivní zostup}
            \subsubsection{Precedenčná syntaktická analýza}
    
        \subsection{Tabuľka symbolov}
        
        \subsection{Vstavané funkcie}
            \noindent
            \textbf{Length}\\
            Vstupným parametrom funkcie je string a výstupným je číselná hodnota dĺžky vstupného stringu.
            \textbf{SubStr}\\
            Vstupné parametre funkcie sú string  a dva integery. Výstupom je podreťazec vstupného
            stringu, ktorého začiatok je určený prvým integerom a jeho dĺžka druhým. Ak je index dĺžky
            0 alebo indexujeme mimo medze stringu návratovou hodnotou je prázdny reťazec. Ak by časť 
            znakov podreťazca pripadala mimo medze základného stringu návratovou hodnotou je string
            obsahujúci iba znaky z medzí vstupného stringu.
            \textbf{Asc}\\
            Vstupné parametre funkcie sú reťazec string a integer. Výstupnou hodnotou je ordinálna
            hodnota (ASCII) znaku v stringu na pozícii zadanej integerom. Ak integer zasahuje mimo 
            medze stringu návratovou hodnotou je 0.
            \textbf{Chr}\\
            Vstupným parametrom je integer. Výstupom je znak, ktorého ASCII kód bol zadaný integerom.
            V prípade integeru momo medzí 0-255 je chovanie funkcie nedefinované.
            
        \subsection{Generovanie kódu}
        
    \section{Rozšírenia}
    Boli implementované celkom 3 rozšírenia
        \begin{itemize}
            \item[*] base - celočíselné konštanty možno zadávať aj v 2, 8 a 16-tkovej sústave.
            \item[*] funexp - volanie funkcie môže byť súčasťou výrazu, zároveň môžu byť výrazy
            v parametroch volaní funkcie.
            \item[*] ifthen - podporuje zjednodušený podmienený príkaz If-Then bez časti Else.
        \end{itemize}
    
    \section{Testovanie}
    Testovanie jednotlivých častí aj celého celku interpretu prebehlo pomocou verejne zdielanej 
    skupiny vyše 450 testov, na ktorých tvorbe sa podieľali ochotní riešitelia tohtoročného 
    zadania projetu do predmetu IFJ. Túto iniciatívu zo strany študentov sme všetci vnímali
    veľmi pozitívne a prospešne pre všetkých riešiteľov projektu. Aj vďaka tomuto množstvu 
    testov sa nám podarilo pokryť takmer celý rozsah interpretu čo nám potvrdili aj výsledky
    z pokusných termínov odovzdania.
    
    \section{Práca v tíme}
    Ako tím sme sa začali schádzať pomerne skoro po zaregistrovaní zadania. Stretnutie sme mali aspoň
    jeden krát týždenne, kde sme prediskutovali aktuálny stav práve implementovaných častí a ďalší 
    postup či korekciu v zdrojovom kóde. Prácu sme rozdeľovali podľa schopností a záujmu jednotlivých
    členov tímu. Po pridelení úlohy sme stanovili dedline, aby sme sa vyhli dlhému čakaniu na 
    implementáciu nadväzujúcej časti projektu. Jednotlivé časti sme sa snažili implementovať paralelne
    s preberanou látkou na prednáškach aby sme čo najskôr využili čerstvo získané informácie.

        \subsection{Správa zdrojového kódu}
        Pre správu a zdielanie zdrojových súborov sme využili webovú službu GitHub s ktorou sme už 
        v priebehu štúdia mali skúsenosti a vedeli sme, že splní svoju úlohu výborne. Ako komunikačný
        kanál sme využili skupinovú konverzáciu v sociálnej sieti Facebook, ktorá nám všetkým vyhovovala.
        Nástenku rozdelenia úloh dokončených, aj čakajúcich na implementáciu nám zabezpečil projektový
        software Trello, ktorý sme taktiež už využili pri riešení skupinového projektu v predošlom
        štúdiu. Taktiež sme vedeli že splní úlohu na ktorú ho potrebujeme. Vďaka tímto informačným 
        kanálom mohli mať všetci členovia tímu prístup k najaktuálnejšej  verzii projektu a reagovať
        na vzniknuté chyby efektívne.

        \newpage
        \subsection{Rozdelenie}
        \noindent
        Bodové rozdelenie je nasledovné:\\ \\
        Tomáš Nereča (xnerec00) \quad 26\% \\
        Samuel Obuch (xobuch00) \quad 26\% \\
        Jiří Vozár (xvozar04) \quad 26\% \\
        Ján Farský (xfarsk00) \quad 22\% \\
        Dôvod nerovnomerného rozdelenia bodov bolo nedostatočné splnenie pridelených častí jednému 
        zo členov tímu, čo viedlo k potrebnému prepisu kódu od ostatných členov tímu.                                 

    \section{Záver}
    S projektom podobného rozsahu sa ešte nikto z nás pred tým nestretol, preto ho pokladáme dobrnú
    skúsenosť pre každého z nás. Pri jeho riešení sme prakticky využili získané informácie z 
    predmetov IFJ a IAL. Pre správne fungovanie tímu bola potrebná dobrá komunikácia, ktorá pretvala
    počas celého prebehu projektu aj vďaka pravidelným stretnutiam. Výsledkom tejto práce je funkčný
    a z nášho pohľadu vydarený interpret jazyka IFJ17.
    
    \section{Prílohy}
        \subsection{LL gramatika}
        \subsection{LL tabuľka}
        \subsection{Precedenčná tabuľka}

\end{document}
