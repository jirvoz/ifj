\documentclass{article}
\usepackage[slovak]{babel}
\usepackage[utf8]{inputenc}

\usepackage{indentfirst}
\usepackage{graphicx}

\begin{document}
    \begin{titlepage}
        \begin{center}
            \vspace{\stretch{0.382}}
            
            \includegraphics[scale=0.4]{FITlogo.png}\\[30mm]
            
            \LARGE
            Implementácia prekladača imperatívneho jazyka IFJ17 \\
            \large
            Dokumentácia k projektu IFJ/IAL\\[55mm]
            
            \begin{tabular}{r l l}
                Tím: & 094 \\
                Varianta: & II \\
                Vedúci: & Tomáš Nereča (xnerec00) & 26\% \\
                Ďalší členovia: & Samuel Obuch (xobuch00) & 26\% \\
                    & Jiří Vozár (xvozar04) & 26\% \\
                    & Ján Farský (xfarsk00) & 22\% \\
                Implementované rozšírenia: & BASE, FUNEXP, IFTHEN\\
            \end{tabular}
            
            \vspace{\stretch{0.618}}
        \end{center}
    \end{titlepage}

    \tableofcontents
    \newpage
    
    \section{Úvod}
        Dokumentácia popisuje implementáciu imperatívneho prekladača jazyka IFJ17. Prekladač načítava
        zdrojový kód jazyka IFJ17 z štandardného vstupu a ten pomocou lexikálnej, syntaktickej a sémantickej
        analýzy vyhodnotí a vygeneruje potrebné inštrukcie v IFJcode17 pre interpret jazyka IFJ17. V prípade
        výskytu chyby pri analýze kódu IFJ17 prekladač vráti kód príslušnej chyby a chybovú hlášku na štandardný
        chybový výstup.  Dokumentácia taktiež zahŕňa zhodnotenie práce v tíme a približuje logiku a fungovanie
        jednotlivých častí prekladača. 

        \subsection{Varianta zadania}
    
    \section{Implementácia}
    
        \subsection{Lexikálna analýza}
        
        \subsection{Syntaktická analýza}
            \subsubsection{Rekurzivní zostup}
            \subsubsection{Precedenčná syntaktická analýza}
    
        \subsection{Tabuľka symbolov}
        
        \subsection{Vstavané funkcie}
            \noindent
            \textbf{Length}\\
            Vstupným parametrom funkcie je string a výstupným je číselná hodnota dĺžky vstupného stringu.\\~\\
            \textbf{SubStr}\\
            Vstupné parametre funkcie sú string  a dva integery. Výstupom je podreťazec vstupného
            stringu, ktorého začiatok je určený prvým integerom a jeho dĺžka druhým. Ak je index dĺžky
            0 alebo indexujeme mimo medze stringu návratovou hodnotou je prázdny reťazec. Ak by časť 
            znakov podreťazca pripadala mimo medze základného stringu návratovou hodnotou je string
            obsahujúci iba znaky z medzí vstupného stringu.\\~\\
            \textbf{Asc}\\
            Vstupné parametre funkcie sú reťazec string a integer. Výstupnou hodnotou je ordinálna
            hodnota (ASCII) znaku v stringu na pozícii zadanej integerom. Ak integer zasahuje mimo 
            medze stringu návratovou hodnotou je 0.\\~\\
            \textbf{Chr}\\
            Vstupným parametrom je integer. Výstupom je znak, ktorého ASCII kód bol zadaný integerom.
            V prípade integeru momo medzí 0-255 je chovanie funkcie nedefinované.
            
        \subsection{Generovanie kódu}
        
    \section{Rozšírenia}
    Boli implementované celkom 3 rozšírenia
        \begin{itemize}
            \item[*]  \textbf{base}   - podpora zadávania celočíselných konštánt v 2, 8 a 16-tkovej sústave.
            \item[*]  \textbf{funexp} - podpora volania funkcií, ktoré môžu v argumentoch obsahovať výrazy 
                                        a zároveň súčasťou výrazu môže byť volanie funkcie.
            \item[*]  \textbf{ifthen} - podpora zjednodušenej konštrukcie podmienky If-Then bez časti Else
                                        a zároveň viacnásobnej konštrukcie Elseif-Then.
        \end{itemize}
    
    \section{Testovanie}
    Na začiatku sme určili jedného člena, ktorý písal jednotkové testy pre lexikálny analyzátor. 
    Neskôr sme napísali zopár vlastných regresných testov no keď sme sa dozvedeli o verejnej 
    databázi testov našich kolegov tak sme do nej prispeli aj vlastnými testami a využili 
    ostatné testy na čo najrobustnejšie otestovanie funkčnosti nášho prekladača.
    
    \section{Práca v tíme}
    Ako tím sme sa začali schádzať pomerne skoro po zaregistrovaní nášho zadania. Stretnutie sme mali 
    minimálne raz týždenne, kde sme prediskutovali aktuálny stav práve implementovaných častí 
    a ďalší postup či korekciu v zdrojovom kóde. Prácu sme sa snažili rozdeľovať rovnomerne medzi 
    všetkých členov tímu čo sa nie vždy darilo. Po pridelení úlohy sme stanovili deadline, aby sme 
    mohli čo najskôr pokračovať na nasledujúcej časti projektu. Jednotlivé časti sme sa snažili 
    implementovať paralelne s preberanou látkou na prednáškach aby sme korektne implementovali
    jednotlivé časti a vyhli sa zbytočným chybám.

        \subsection{Správa zdrojového kódu}
        Pre správu a zdielanie zdrojových súborov sme využili webovú službu GitHub, ktorú sme už 
        počas štúdia využili na verzovanie projektov v iných predmetoch. Ako komunikačný kanál sme 
        prevažne využívali skupinovú konverzáciu na sociálnej sieti Facebook, ktorá nám všetkým 
        vyhovovala. Na sledovanie postupu pri plnení pridelených úloh a oznamovanie nájdených chýb 
        sme využívali online nástroj Trello, ktorý slúži ako online kanban pre sledovanie projektov. 
        Vďaka týmto informačným kanálom mohli mať všetci členovia tímu prístup k najaktuálnejšej
        verzii projektu a reagovať na vzniknuté chyby efektívne.

        \newpage
        \subsection{Rozdelenie}
        \noindent
        Bodové rozdelenie je nasledovné:\\~\\
        \begin{tabular}{l l}
            Tomáš Nereča (xnerec00) & 26\% \\
            Samuel Obuch (xobuch00) & 26\% \\
            Jiří Vozár (xvozar04)   & 26\% \\
            Ján Farský (xfarsk00)   & 22\% \\~\\
        \end{tabular} \\
        Dôvod nerovnomerného rozdelenia bodov bolo nedostatočné splnenie pridelených častí jednému 
        z členov tímu, čo viedlo k potrebnému prepisu zdrojového kódu ostatnými členmi tímu.                                 

    \section{Záver}
    S projektom podobného rozsahu sa ešte nikto z nás predtým nestretol, preto ho pokladáme za dobrú
    skúsenosť pre každého z nás. Pri jeho riešení sme prakticky využili získané vedomosti z predmetov 
    IFJ a IAL. Pre správne fungovanie tímu bola potrebná dobrá komunikácia, ktorá pretvala počas celého
    priebehu projektu či už vďaka pravidelným stretnutiam alebo využitým nástrojom ale aj individuálne
    schopnosti jednotlivých členov. Výsledkom tejto práce je funkčný a z nášho pohľadu vydarený 
    prekladač jazyka IFJ17.
    
    \section{Prílohy}
        \subsection{LL gramatika}
        \subsection{LL tabuľka}
        \subsection{Precedenčná tabuľka}

\end{document}